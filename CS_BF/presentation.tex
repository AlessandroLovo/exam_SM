\documentclass{beamer}
\usepackage[italian]{babel}
\usepackage[utf8]{inputenc}
\usepackage{import}
\usepackage{float}
\usepackage{subfigure}
\usepackage{amssymb,amsmath,amsthm,amsfonts}
% \usepackage{tabularx}
% \usepackage{indentfirst}
% \usepackage{wrapfig,booktabs}
%\usepackage[small]{caption}
% \usepackage{subcaption}
% \usepackage{eucal}
% \usepackage{eso-pic}
\usepackage{hyperref}
\usepackage{url}
\usepackage{booktabs}
% \usepackage{afterpage}
% \usepackage{parskip}
\usepackage{listings}
\usepackage{fancyhdr}
\usepackage{textcomp}
\usepackage{cite}
\usepackage{multirow,multicol}
% \usepackage{setspace}
\usepackage[version=4]{mhchem}
\usepackage{nicefrac}
\usepackage{siunitx}

\usepackage{caption}
\captionsetup{tableposition=top,font=small,width=0.8\textwidth}
%\usepackage[table]{xcolor}
\usepackage[arrowdel]{physics}
\usepackage{mathtools}
\usepackage{tablefootnote}
\usepackage{enumitem}

\setlist[description]{font={\scshape}} %style=unboxed,style=nextline
\usepackage{floatflt}
\usepackage{commath}
\usepackage{bm}
\usepackage{ifthen}
\usepackage{comment}
% \usepackage[colorinlistoftodos,textsize=tiny]{todonotes}

\newcommand{\overbar}[1]{\mkern 1.5mu\overline{\mkern-1.5mu#1\mkern-1.5mu}\mkern 1.5mu}
\let\oldfrac\frac
\renewcommand{\frac}[3][d]{\ifthenelse{\equal{#1}{d}}{\oldfrac{#2}{#3}}{\nicefrac{#2}{#3}}}


\usetheme{Madrid}
\setbeamercovered{transparent}
\useoutertheme{miniframes}

\title[Stati coerenti di materia]{Stati coerenti per bosoni e fermioni}
\author{Alessandro Lovo}

\begin{document}

\begin{frame}
  \maketitle
\end{frame}

\section{Seconda quantizzazione della materia}
\begin{frame}
  \frametitle{Luce e materia}
  Consideriamo l'hamiltoniana in prima quantizzazione $\hat{H}_{cl} = -\frac{\hbar^2}{2m}\nabla^2 + U(\mathbf{r})$ con autostati
  \begin{equation*}
    \hat{H}_{cl} \ket{\phi_{\alpha}} = E_{\alpha} \ket{\phi_{\alpha}}
  \end{equation*}
  Una generica funzione d'onda può dunque essere espressa come $\ket{\psi} = \sum_{\alpha} c_{\alpha} \ket{\phi_{\alpha}}$ e si può dunque scrivere l'energia associata come
  \begin{equation*}
    H = \bra{\psi} \hat{H}_{cl} \ket{\psi} = \sum_{\alpha} \frac{E_{\alpha}}{2} \left(c_{\alpha}c_{\alpha}^* + c_{\alpha}^*c_{\alpha} \right)
  \end{equation*}
  Con un procedimento simile a quello per la seconda quantizzazione del campo elettromagnetico è possibile promuovere ad operatori $c_{\alpha}, c_\alpha^*$ e di conseguenza $H, \psi$.

\end{frame}

\begin{frame}
  \frametitle{Operatori bosonici e fermionici}
  \begin{gather*}
    \hat{\psi} = \sum_{\alpha} \hat{c}_{\alpha} \phi_{\alpha}\\
    \hat{N}_{\alpha} = \hat{c}_{\alpha}^\dagger \hat{c}_\alpha \\
    \hat{H} = \sum_{\alpha} E_{\alpha} \left( \hat{N}_{\alpha} + \frac{1}{2} \right) \approx \sum_{\alpha} E_{\alpha} \hat{N}_{\alpha}
  \end{gather*}
  Nel caso di bosoni
  \begin{gather*}
    [\hat{c}_{\alpha},\hat{c}_{\beta}^\dagger] = \delta_{\alpha\beta}, \,
    [\hat{c}_{\alpha},\hat{c}_{\beta}] = [\hat{c}_{\alpha}^\dagger,\hat{c}_{\beta}^\dagger] = 0, \,
    \sigma(\hat{N}_\alpha) = \mathbb{N}
  \end{gather*}
  mentre nel caso di fermioni
  \begin{gather*}
    \{\hat{c}_{\alpha},\hat{c}_{\beta}^\dagger\} = \delta_{\alpha\beta}, \,
    \{\hat{c}_{\alpha},\hat{c}_{\beta}\} = \{\hat{c}_{\alpha}^\dagger,\hat{c}_{\beta}^\dagger\} = 0, \,
    \sigma(\hat{N}_\alpha) = \{0,1\}
  \end{gather*}
\end{frame}

\begin{frame}
  \frametitle{Stati numero e stati coerenti}
  Come nel caso della luce gli operatori $\hat{c}_{\alpha},\hat{c}_{\alpha}^\dagger$ rispettivamente distruggono e creano una particella nello stato $\phi_\alpha$ agendo sullo spazio di Fock in rappresentazione di numero:
  \begin{equation*}
    \hat{N}_\alpha \ket{\dots n_\alpha \dots} = n_\alpha \ket{\dots n_\alpha \dots}
  \end{equation*}
  Viene dunque spontaneo chiedersi se anche per la materia sia possibile avere degli stati coerenti, ossia autostati dell'operatore di annichilazione in cui il numero di particelle non sia fissato
  \begin{equation*}
    \hat{c}_\alpha \ket{c_\alpha} = c_\alpha \ket{c_\alpha}
  \end{equation*}
  % \begin{gather*}
  %   \hat{c}_{\alpha} \ket{\dots n_\alpha \dots} = \sqrt{n_\alpha} \ket{\dots n_\alpha - 1 \dots} \\
  %   \hat{c}_{\alpha}^\dagger \ket{\dots n_\alpha \dots} = \sqrt{n_\alpha + 1} \ket{\dots n_\alpha + 1 \dots}
  % \end{gather*}
\end{frame}

\section{Stati coerenti}
\begin{frame}
  \frametitle{Bosoni}
  Nel caso di particelle bosoniche non c'è limite al numero di particelle che occupino un singolo stato e infatti con la condensazione di Bose-Einstein occupano tutte lo stato ad energia inferiore $\phi_0$. Inoltre poichè le regole di commutazione sono le stesse della luce è immediato esprimere lo stato coerente come
  \begin{equation*}
    \ket{c}_\alpha = e^{-\frac[f]{|c_\alpha|^2}{2}} \sum_{n_\alpha = 0}^\infty \frac{c_\alpha}{\sqrt{n_\alpha!}} \ket{n_\alpha}
  \end{equation*}
  dove $\overbar{N} = |c_\alpha|^2$ rappresenta il numero medio di bosoni nello stato $\phi_\alpha$.\\
  Analogamente al caso della luce la fluttuazione di $\hat{\psi}$ (che corrisponde al campo elettrico) calcolata sullo stato coerente è la stessa di quella sullo stato di vuoto.
\end{frame}

\begin{frame}
  \frametitle{Fermioni}
  Nel caso di fermioni la situazione è più complicata poiché ogni stato $\phi_\alpha$ può contenere al massimo una particella. Questo altro non è che il principio di esclusione di Pauli che in seconda quantizzazione può essere formulato come $(\hat{c}_\alpha^\dagger)^2 \ket{0} = 0$.\\
  Ciononostante è comunque possibile definire lo stato coerente come autosatato dell'operatore di annichilazione
  \begin{gather*}
    \hat{c}_\alpha \ket{c_\alpha} = c_\alpha \ket{c_\alpha}, \,
    \bra{c_\alpha} \hat{c}_\alpha^\dagger  = \overbar{c}_\alpha \bra{c_\alpha}
  \end{gather*}
  Tuttavia a causa delle regole di anticommutazione $c_\alpha,\overbar{c}_\alpha$ non sono numeri complessi bensì numeri di Grassmann.
  \begin{gather*}
    c_\alpha^2 = \overbar{c}_\alpha^2 = 0, \, c_\alpha\overbar{c}_\alpha + \overbar{c}_\alpha c_\alpha = 1
  \end{gather*}
\end{frame}

\begin{frame}
  \frametitle{Campi 'classici'}
  Gli stati coerenti possono essere usati come ponte per ottenere l'analogo 'classico' del campo di materia $\hat{\psi}$:
  \begin{gather*}
    \hat{\psi} = \sum_\alpha \phi_\alpha \hat{c}_\alpha \, \to \, \psi = \sum_\alpha \phi_\alpha c_\alpha\\
    \hat{\psi}\ket{\psi} = \psi\ket{\psi}\\
    \ket{\psi} = \bigotimes_\alpha \ket{c_\alpha}, \, \hat{c}_\alpha \ket{c_\alpha} = c_\alpha \ket{c_\alpha}
  \end{gather*}
  Nel caso di bosoni $\psi$ sarà un campo complesso, mentre nel caso di fermioni un campo di Grassmann.
\end{frame}




\end{document}
