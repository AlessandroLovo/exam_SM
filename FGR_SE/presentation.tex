\documentclass{beamer}
\usepackage[italian]{babel}
\usepackage[utf8]{inputenc}
\usepackage{import}
\usepackage{float}
\usepackage{subfigure}
\usepackage{amssymb,amsmath,amsthm,amsfonts}
% \usepackage{tabularx}
% \usepackage{indentfirst}
% \usepackage{wrapfig,booktabs}
%\usepackage[small]{caption}
% \usepackage{subcaption}
% \usepackage{eucal}
% \usepackage{eso-pic}
\usepackage{hyperref}
\usepackage{url}
\usepackage{booktabs}
% \usepackage{afterpage}
% \usepackage{parskip}
\usepackage{listings}
\usepackage{fancyhdr}
\usepackage{textcomp}
\usepackage{cite}
\usepackage{multirow,multicol}
% \usepackage{setspace}
\usepackage[version=4]{mhchem}
\usepackage{nicefrac}
\usepackage{siunitx}

\usepackage{caption}
\captionsetup{tableposition=top,font=small,width=0.8\textwidth}
%\usepackage[table]{xcolor}
\usepackage[arrowdel]{physics}
\usepackage{mathtools}
\usepackage{tablefootnote}
\usepackage{enumitem}

\setlist[description]{font={\scshape}} %style=unboxed,style=nextline
\usepackage{floatflt}
\usepackage{commath}
\usepackage{bm}
\usepackage{ifthen}
\usepackage{comment}
% \usepackage[colorinlistoftodos,textsize=tiny]{todonotes}

\newcommand{\overbar}[1]{\mkern 1.5mu\overline{\mkern-1.5mu#1\mkern-1.5mu}\mkern 1.5mu}
\let\oldfrac\frac
\renewcommand{\frac}[3][d]{\ifthenelse{\equal{#1}{d}}{\oldfrac{#2}{#3}}{\nicefrac{#2}{#3}}}


\usetheme{Madrid}
\setbeamercovered{transparent}
\useoutertheme{miniframes}

\title[ROF \& ES]{Regola d'Oro di Fermi ed Emissione Spontanea}
\author{Alessandro Lovo}

\begin{document}

\begin{frame}
  \maketitle
\end{frame}

\section{Regola d'Oro di Fermi (ROF)}
\begin{frame}
  \frametitle{Curiosità}
  La maggior parte del lavoro che condusse alla regola d'oro venne in realtà svolto da Paul Dirac che giunse nel 1926 ad una formula finale praticamente identica.
  Più tardi Fermi comprese l'importanza di tale formula tanto da fornirle l'appellativo di \emph{Regola d'oro}

\end{frame}

\begin{frame}
  \frametitle{Formulazione}
  Consideriamo un sistema descritto da un hamiltoniana

  \begin{equation*}
    \hat{H} = \hat{H}_0 + \hat{H}_I
  \end{equation*}

  dove $H_I$ rapresenta una piccola perturbazione.\\
  Allora il tasso di transizione $W_{I \to F}$ tra gli autostati di $\hat{H}_0$ $\ket{I}$, $\ket{F}$ è dato da:

  \begin{equation*}
    W_{I \to F} = \frac{2\pi}{\hbar} \left| \bra{I} \hat{H}_I \ket{F} \right|^2 \delta(E_F - E_I)
  \end{equation*}

  Dove la delta di Dirac assicura la conservazione dell'energia.
\end{frame}

\section{Emissione spontanea}
\begin{frame}
  \frametitle{Il problema}
  Consideriamo ora un atomo di idrogeno interagente con la radiazione elettromagnetica (e.m.):
  \begin{equation*}
    \hat{H} \approx \hat{H}_{matt} + \hat{H}_{rad} + \hat{H}_D = \hat{H}_0 + \hat{H}_D
  \end{equation*}
  L'hamiltoniana di perturbazione è l'approssimazione di dipolo dell'interazione:
  \begin{equation*}
    \hat{H}_D = \frac{e}{m} \hat{\mathbf{A}} \cdot \hat{\mathbf{p}}
  \end{equation*}
  mentre l'hamiltoniana impertubata è in prima quantizzazione per l'atomo e in seconda quantizzazione per la radiazione e.m. con autostati rispettivamente
  \begin{gather*}
    \hat{H}_{matt}\ket{a} = E_a \ket{a}, \quad \hat{H}_{rad}\bigotimes_{\mathbf{k}s} \ket{n_{\mathbf{k}s}} =
    \left(\sum_{\mathbf{k}s}\hbar\omega_{k}n_{\mathbf{k}s} \right) \bigotimes_{\mathbf{k}s} \ket{n_{\mathbf{k}s}}
  \end{gather*}
\end{frame}

\begin{frame}
  \frametitle{Stato iniziale e finale}
  L'emissione spontanea è la transizione tra uno stato eccitato $\ket{b}$ ad uno ad energia inferiore $\ket{a}$ con la creazione di un fotone dallo stato di vuoto del campo e.m., dunque
  \begin{gather*}
    \ket{I} = \ket{b}\ket{0} \quad
    \ket{F} = \ket{a}\ket{1_{\mathbf{k}s}}
  \end{gather*}
  Esprmendo il potenziale vettore in approssimazione di dipolo in termini degli operatori di creazione e distruzione si può procedere ad applicare la regola d'oro di Fermi.
  \begin{equation*}
    \hat{\mathbf{A}} = \sum_{\mathbf{k}s} \sqrt{\frac{\hbar}{2\epsilon_0\omega_kV}} \left(\hat{a}_{\mathbf{k}s} + \hat{a}_{\mathbf{k}s}^\dagger \right) \epsilon_{\mathbf{k}s}
  \end{equation*}
\end{frame}

\begin{frame}
  \frametitle{}
  \begin{align*}
    \bra{I} \hat{H}_I \ket{F} & =  \frac{e}{m} \bra{b}\bra{0} \sum_{\mathbf{k'}s'} \sqrt{\frac{\hbar}{2\epsilon_0\omega_{k'}V}}
    \left(\hat{a}_{\mathbf{k'}s'} + \hat{a}_{\mathbf{k'}s'}^\dagger \right) \epsilon_{\mathbf{k'}s'} \cdot \hat{\mathbf{p}} \ket{a}\ket{1_{\mathbf{k}s}}\\
     & = \frac{e}{m} \sqrt{\frac{\hbar}{2\epsilon_0\omega_{k}V}} \epsilon_{\mathbf{k}s} \cdot \bra{b} \hat{\mathbf{p}} \ket{a}
  \end{align*}

  \begin{align*}
    W_{ba,\mathbf{k}s}^{spont} & :=  W_{I \to F} = \frac{2\pi}{\hbar} \left| \bra{I} \hat{H}_I \ket{F} \right|^2 \delta(E_F - E_I) \\
    & = \frac{2\pi}{\hbar} \left(\frac{e}{m}\right)^2 \frac{\hbar}{2\epsilon_0\omega_{k}V}
    \left| \epsilon_{\mathbf{k}s} \cdot \bra{b} \hat{\mathbf{p}} \ket{a} \right|^2 \delta(E_b - E_a - \hbar\omega_k)
  \end{align*}
\end{frame}

\begin{frame}
  \frametitle{Elemento di dipolo}
  Dall'equazione di Heisenberg è possibile esprimere il momento in termini della posizione:
  \begin{equation*}
    \hat{\mathbf{p}} = m \frac{d \mathbf{r}}{dt} = m \frac{[\mathbf{r}, \hat{H}_{matt}]}{i\hbar}
  \end{equation*}
  E di conseguenza
  \begin{equation*}
    \bra{a} \hat{\mathbf{p}} \ket{b} = \frac{m}{i\hbar}\bra{a} \mathbf{r}\hat{H}_{matt} - \hat{H}_{matt}\mathbf{r} \ket{b}
    = m\frac{E_b - E_a}{i\hbar} \bra{a} \mathbf{r} \ket{b}
  \end{equation*}
  Se ora definiamo $\omega_{ba} = \frac{E_b - E_a}{\hbar}$ e $\mathbf{d}_{ba} = \bra{a} -e\mathbf{r} \ket{b}$ possiamo scrivere
  \begin{equation*}
    W_{ba,\mathbf{k}s}^{spont} = \frac{\pi \omega_{ba}^2}{\epsilon_0\omega_kV} \left|\epsilon_{\mathbf{k}s} \cdot \mathbf{d}_{ba} \right|^2
    \delta(\hbar \omega_k - \hbar \omega_{ba})
  \end{equation*}
\end{frame}

\begin{frame}
  \frametitle{Tasso totale di emissione spontanea}
  Integrando su tutti i possibili stati finali per il fotone emesso si ottiene
  \begin{align*}
    W_{ba}^{spont} & = \sum_{\mathbf{k}s} W_{ba,\mathbf{k}s}^{spont} = \frac{V}{(2\pi)^3}\int d^3k \sum_s W_{ba,\mathbf{k}s}^{spont}\\
    & = \frac{1}{8\pi^2\epsilon_0} \int_0^{\infty} k^2 dk \frac{\omega_{ba}^2}{\omega_k} \delta(\hbar \omega_k - \hbar \omega_{ba})
    \int d\Omega \sum_s \left|\epsilon_{\mathbf{k}s} \cdot \mathbf{d}_{ba} \right|^2
  \end{align*}
  % \begin{align*}
  %   W_{ba}^{spont} & = \sum_{\mathbf{k}s} W_{ba,\mathbf{k}s}^{spont} = \frac{V}{(2\pi)^3}\int d^3k \sum_s W_{ba,\mathbf{k}s}^{spont}\\
  %   & = \frac{1}{8\pi^2\epsilon_0} \int_0^{\infty} k^2 dk \int d\Omega \frac{\omega_{ba}}{ck}
  %   \frac{\delta(k - \frac[f]{\omega_{ba}}{c})}{\hbar c} \sum_s \left|\epsilon_{\mathbf{k}s} \cdot \mathbf{d}_{ba} \right|^2 \\
  %   & = \frac{\omega_{ba}^3}{8\pi^2\epsilon_0\hbar c^3} \int_0^{\pi} \sin\theta d\theta \int_0^{2\pi} d\varphi \sum_s \left|\epsilon_{\mathbf{k}s} \cdot \mathbf{d}_{ba} \right|^2
  % \end{align*}

\end{frame}

\begin{frame}
  \frametitle{Integrazione angolare}
  Essendo i due versori di polarizzazione $\epsilon_{\mathbf{k}s}$ ortogonali tra loro e ortogonali a $\mathbf{k}$, $\left\{\mathbf{n}=\frac[f]{\mathbf{k}}{k}, \epsilon_{\mathbf{k}1}, \epsilon_{\mathbf{k}2}\right\}$ formano una base ortogonale. Scegliendo $\mathbf{d}_{ba}$ come asse per le coordinate polari dell'integrazione si ottiene
  \begin{align*}
    \sum_s \left|\epsilon_{\mathbf{k}s} \cdot \mathbf{d}_{ba} \right|^2 = \left| \mathbf{d}_{ba} \right|^2 - \left|\mathbf{n}\cdot \mathbf{d}_{ba}\right|^2
    =  \left| \mathbf{d}_{ba} \right|^2 (1 - \cos^2\theta)
  \end{align*}
  da cui
  \begin{equation*}
    \int d\Omega \sum_s \left|\epsilon_{\mathbf{k}s} \cdot \mathbf{d}_{ba} \right|^2
    = \int_0^{\pi} \sin\theta d\theta \int_0^{2\pi} d\varphi \left| \mathbf{d}_{ba} \right|^2 \sin^2\theta
    = \frac{8\pi}{3} \left| \mathbf{d}_{ba} \right|^2
  \end{equation*}
\end{frame}

\begin{frame}
  \frametitle{Integrazione radiale}
  Considerando che $\omega_k = ck$ è possibile giungere alla formula finale:
  \begin{align*}
    W_{ba}^{spont} & = \frac{1}{3\pi\epsilon_0} \left| \mathbf{d}_{ba} \right|^2 \int_0^{\infty} k^2 dk \frac{\omega_{ba}}{ck} \frac{\delta(k - \frac[f]{\omega_{ba}}{c})}{\hbar c}\\
    & = \frac{\omega_{ba}^3}{3\pi\epsilon_0\hbar c^3} \left| \mathbf{d}_{ba} \right|^2
  \end{align*}
\end{frame}


\end{document}
